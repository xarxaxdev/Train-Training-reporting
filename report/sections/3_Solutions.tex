
\section{Solutions}
\color{green}
\subsection{Definitions}
To not reiterate the descriptions of the example files we are using for our properly-structured solution, I will be describing it here. While our solution is graph-based; it should be valid for other approaches. The respective files are:
\begin{itemize}
	\item  \textbf{input.lp} File that generates the base problem representation (a graph or other); with no choice rules.
	\item  \textbf{path.lp} File that uses choice rules to determine the path.
	\item  \textbf{output.lp} File that deduces, given the elected path, which action should be followed by each train; with no choice rules.
	\item  \textbf{malfunctions.lp} File that loads (selectively, according to the timestamp) and saves upon malfunction. Also where path implications from a malfunction should be deduced; with no choice rules.
\end{itemize}
\color{black} \color{gray} I would get rid of this. I think our structure is not important, as it adds nothing to the execution, for the order of lines does not matter. \color{black}
\color{blue}
\begin{itemize}
	\item make small section for it
	\item flatland and action saving and state machine problems
\end{itemize}
\color{black}



\subsection{Naive (vertex with directions) file partition}
malfunctions + main
\color{blue}
\begin{itemize}
	\item motivation (comparison of passing and recomputing)
	\item implementation
	\begin{itemize}
		\item just adding malfunction handling on top
		\item passing just basic information
	\end{itemize}
	\item implications
	\begin{itemize}
		\item slower grounding?
		\item else similar as just reschedule
	\end{itemize}
\end{itemize}
\color{black}


\subsection{Smart (vertex with directions) file partition}
\color{green} Would make this one the first, as it is the easiest to explain. And easy to narrate as we implemented it. \color{black}
malfunctions + input+ path+output
\color{blue}
\begin{itemize}
	\item motivation
	\item implementation
	\begin{itemize}
		\item short summary on graph structure
		\item talk about split (whats primary, whats secondary)
		\item extending the window (worst case assumption)
	\end{itemize}
	\item implications
	\begin{itemize}
		\item escalating window
		\item bad scaling
		\item quick with good estimates
	\end{itemize}
\end{itemize}
\color{black}

\subsection{Incremental}
\color{blue}
\begin{itemize}
	\item motivation (better window)
	\item implementation
	\begin{itemize}
		\item work clingo side
		\item solving adaptations
		\item step splitting
		\item primary secondary split
	\end{itemize}
	\item implications
	\begin{itemize}
		\item sometimes slower
		\item faster overall
		\item more adaptive (theoretically does not require horizon)
	\end{itemize}
\end{itemize}
\color{black}

