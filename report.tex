\documentclass{llncs}
\usepackage{xcolor}
\usepackage{tikz}
\usetikzlibrary{arrows.meta, positioning} 

\begin{document}
\title{Train Scheduling and Malfunctions in ASP}
\author{Guillem Gili i Bueno, Felix Marc Kratzsch}
\institute{Potsdam University}
\date{\today}
\maketitle
\begin{multicols*}{2}


\section{Abstract}

\section{Introduction}
\begin{itemize}
    \item introduce topic
    \item what doe we focus on?
    \item how are we gonna work with it?
    \item overview over the paper? (small table of contents?)
\end{itemize}

\section{Explanation Flatland-ASP?}
\begin{itemize}
    \item What is Flatland? (What aspects do we focus on?)
    \item What is ASP (How do we use it?)
\end{itemize}
\section{Approaches}

%TODO: Add Figure With transition
The basic concept is, to get away from the flatland representation and instead compute the problem on a graph representation. The main difficulty lies therein, that in Flatland Cells, are not the only thing constraining movement, but additionally a direction needs to either be tracked as additional information, or be worked into the graph. For this encoding we chose the second approach. To ensure a graph-like structure vertices are not defined by just the cell, but by the cell the train came from. This leads to a graph with directed edges and paths modelling the environment.
\color{blue}
\begin{itemize}
    \item motivation
    \item implementation
    \item implications
    \begin{itemize}
        \item shortcomings
        \item why Simplified
        \item why incremental
    \end{itemize}
\end{itemize}

\subsubsection*{Naive}
\begin{itemize}
    \item might better be moved to evaluation
\end{itemize}
\color{black}

\subsection{Graphs Simplified}
\begin{itemize}
    \item motivation
    \item implementation
    \item implications (shortcomings strengths)
\end{itemize}

\color{blue}
\section{incremental}
\begin{itemize}
    \item motivation
    \item implementation
    \item implications (shortcomings strengths)
\end{itemize}
\color{black}

\color{blue}
\section{Testing}
\begin{itemize}
    \item Environment (how, why)
    \item Benchmark (how, why)
\end{itemize}
\color{black}

\section{Evaluation}
\begin{itemize}
    \item hypothesis checking from implications
    \item overall time
    \item overall time of reruns
    \item space comsumption (maybe)
\end{itemize}
    



\end{multicols*}
\end{document}