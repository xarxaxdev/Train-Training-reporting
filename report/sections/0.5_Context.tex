\section{Introduction}

This report is a product of the course Railway Scheduling at Potsdam University. It focuses on finding timetables in a Railway Environment. The goal of the course is stated as `pushing the boundary of Flatland`. In particular, the goal of our group this semester was to explore further what concepts were necessary or useful regarding malfunctions within the Flatland Framework.

In this report, a brief introduction to Flatland is made and the particularities surrounding the usage of malfunctions are highlighted. Afterwards we will propose solutions (which are our main contributions to the project), to solve such problems, starting with a baseline and followed by two derivations. One will serve to compare reutilization and recomputation, the other will implement an incremental ASP approach.

As part of the project, we also selected interesting instances, for diverse and comparable scenarios. Our solutions will than be tested against them. It will be followed by a short evaluation, a conclusion and an outlook over the field.