% - malfunctions
% - particularitiews
% - edge cases
% - implications (window, modularity)
% - naive implementation
\section{Malfunctions implications}

\begin{frame}
	\frametitle{Malfunction implications: Edge cases}
		\begin{itemize}
		\item \textbf{Collisions} Train A malfunctioning may affect a Train B.
		\item \textbf{Auto spawning}  Off-map malfunctions may force trains to spawn
		\item \textbf{Some state-related complications}  
	\end{itemize}
	
	
\end{frame}


\begin{frame}
	\frametitle{Malfunction implications: Modularity}
	\textbf{Modularity in Clingo} We can selectively ground/load
		\begin{itemize}
		\item \textbf{Check statement-specific cost}  By selectively loading
		\item \textbf{Generating vs Reading} We study this 
		\item \textbf{Early-Grounding and forwarding} Outside our project  
	\end{itemize}
\end{frame}


\begin{frame}
	\frametitle{Malfunction implications: Window of grounding}
		\textbf{Window-focused representation} 
	\begin{itemize}
		\item \textbf{Ignore prior timestamps to last malfunction} 
		\item \textbf{Dynamically push back horizons} 
		\item \textbf{Currently disallowed} Flatland requires previous actions to match current solution
		
	\end{itemize}
\end{frame}

\section{Implementations}

\begin{frame}
	\frametitle{Implementations 1 \& 2: Directed graph}
	\textbf{Naive vs. Proper setup} 
	\begin{itemize}
		\item \textbf{Each vertex holds its coordinates and the adjacent vertexes} 
		\item \textbf{Implements shared resources} To prevent sharing edges/vertex. 
		\item \textbf{Test Staticly reading vs Generating}  Saving our graph vs cells 		
	\end{itemize}
\end{frame}

