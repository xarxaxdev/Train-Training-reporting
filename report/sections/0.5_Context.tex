\section{Introduction}

This report is a product of the course Railway Scheduling at Potsdam University. It focuses on finding timetables in a Railway Environment. The goal of the course is stated as `pushing the boundary of Flatland`. In particular, the goal of our group this semester was to explore further what concepts were necessary or useful regarding malfunctions within the Flatland Framework.

In this report, a brief introduction to Flatland is made and the particularities surrounding the usage of malfunctions are highlighted. Afterwards we will propose solutions (which are our main contributions to the subject), to solve such problems, starting with a baseline and followed by two derivations. One will serve to compare reutilization and recomputation, the other will implement an incremental ASP approach.

These solutions will than be tested against selected instances, which we deemed interesting followed by a short evaluation, a conclusion and an outlook over the field.

\color{red}
Our contribution is a series of rather challenging and varied test instances+  
Solutions that would work correctly under any malfunction environment
\color{black}
\color{green} Is it fine now? \color{black}