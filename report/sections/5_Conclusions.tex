\section{Conclusions}
\color{blue}
\begin{itemize}
	\item check hypothesis from encodings
	\item some overall stats (might do later) (space, time, ...)
\end{itemize}
\color{black}

\color{green}
\begin{itemize}
	\item research options
	\item optimality, priority, resilience (need for malfunction handling)
\end{itemize}
\color{black}
\color{green}
\begin{itemize}
	\item  \textbf{Evaluate for optimal solutions} Add a prioritize statement, that ensures trains arrive with as much free time as possible . After a malfunction, the pruned solutions are likely to be the suboptimal.(less efficiency, more resilience)
	\item  \textbf{Evaluate for quick, scalable solutions} Do not do any prioritization; since the solution space is very big. Find solutions as they appear. This may seem naive, but in a real-world case scenario, where the horizon of actions can be infinite(and the time value span much bigger)  there is a good case for this to be a priority. (more (inmediated)efficiency, less resilience )
\end{itemize}
\color{black}