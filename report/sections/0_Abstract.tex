\begin{abstract}
<<<<<<< HEAD
The goal of the course is stated as `pushing the boundary of Flatland`.In particular, the goal of our group this semester was to explore further what concepts were necessary or useful regarding malfunctions within the Flatland Framework. In this report, a brief introduction to Flatland is made and the particularities surrounding the usage of malfunctions are highlighted. Then the sets of environments that were found the most interesting for the sake of benchmarking are explained, followed by the solutions that we decided to test for these environments. A comparison is drawn between using solutions that are more reutilization friendly than others. Finally, the results are summarized and  we conclude that approaches with an incremental approach on time should be favored over others, and that no relevant difference exists between reading the map representation and calculating it in runtime.
\color{red}
Our contribution is a series of rather challenging and varied test instances+  
Solutions that would work correctly under any malfunction environment
\color{black}

=======
The goal of the course is stated as `pushing the boundary of Flatland`.In particular, the goal of our group this semester was to explore further what concepts were necessary or useful regarding malfunctions within the Flatland Framework. In this report, a brief introduction to Flatland is made and the particularities surrounding the usage of malfunctions are highlighted. Then the sets of environments that were found the most interesting for the sake of benchmarking are explained, followed by the solutions that we decided to test for these environments. A comparison is drawn between using solutions that are more reutilization friendly than others. Finally, the results are summarized and  we conclude 
\color{green} BLABLALABLBA CONCLUSIONS \color{black}
>>>>>>> 71fadc3 (small nitpicks)
\keywords{ASP \and Multi-Agent Pathfinding \and Resilience planning.}
\end{abstract}