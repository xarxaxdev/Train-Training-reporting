<<<<<<< HEAD
\section{Introduction}
The problem of train-scheduling attempts to model and find valid pathing for a series of train agents that share the same tracks and cannot collide. 

In particular, the problem modeled is a variant of Multi-Agent Pathfinding. To quote from Wikipedia:
\begin{quote}
\emph{The problem of Multi-Agent Pathfinding (MAPF) is an instance of multi-agent planning and consists in the computation of collision-free paths for a group of agents from their location to an assigned target. It is an optimization problem, since the aim is to find those paths that optimize a given objective function, usually defined as the number of time steps until all agents reach their goal cells. MAPF is the multi-agent generalization of the pathfinding problem, and it is closely related to the shortest path problem in the context of graph theory.}
\end{quote}


=======
\section{Context}

Subject context blablabla
>>>>>>> 71fadc3 (small nitpicks)


% TODO
% make small section for it
% – flatland and action saving and state machine problems
